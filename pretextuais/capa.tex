% ---
% Informações de dados para CAPA e FOLHA DE ROSTO
% ---
\instituicao{UNIVERSIDADE TECNOLÓGICA FEDERAL DO PARANÁ} % nome da instituicao
\programa{PROGRAMA DE PÓS-GRADUAÇÃO EM ENGENHARIA MECÂNICA E DE MATERIAIS} % nome do programa
\area{Engenharia de Manufatura} % 

\documento{Dissertação} % [Dissertação] ou [Tese]
\nivel{Mestrado} % [Mestrado] ou [Doutorado]
\titulacao{Mestre} % [Mestre] ou [Doutor]

\titulo{\MakeUppercase{Título para a Tese ou Dissertação}} % titulo do trabalho em portugues
\title{\MakeUppercase{Title for Thesis or Dissertation}}
\autor{Seu nome completo} % autor do trabalho
\cita{Colocar modo de citar seu nome} % sobrenome (maiusculas), nome do autor do trabalho

\palavraschave{Palavra-chave 1; Palavra-chave 2; ...; Palavra-chave n} % palavras-chave do trabalho
\keywords{Keywords 1; Keywords 2; ...; Keywords n} % palavras-chave do trabalho em ingles
\local{Curitiba}
\data{2018}

\orientador{Nome do orientador} % nome do orientador do trabalho
%\orientador[Orientadora:]{Nome da Orientadora} % <- no caso de orientadora, usar esta sintaxe
%\coorientador{Nome do Co-orientador} % nome do co-orientador do trabalho, caso exista
%\coorientador[Co-orientadora:]{Nome da Co-orientadora} % <- no caso de co-orientadora, usar esta sintaxe
%\coorientador[Co-orientadores:]{Nome do Co-orientador} % no caso de 2 co-orientadores, usar esta sintaxe
%\coorientadorb{Nome do Co-orientador 2}	% este comando inclui o nome do 2o co-orientador



\comentario{\UTFPRdocumentodata\ apresentada ao \UTFPRprogramadata\ da \ABNTinstituicaodata\, como requisito parcial para obtenção do título de \UTFPRtitulacaodata\ em Engenharia Mecânica – Área de concentração: \UTFPRareadata.}
% ---
